\documentclass[11pt]{beamer}
\usepackage[utf8]{inputenc}
\usepackage[T1]{fontenc}
\usepackage{lmodern}
\usepackage[ngerman]{babel}
\usetheme{Darmstadt}

\usepackage[shortlabels]{enumitem}

\usepackage{tikz}
\usetikzlibrary{shapes,positioning}



\author{Jens Fischer}
\title{Bildungswesen}
\subtitle{Using Beamer}
\begin{document}
	%\subtitle{}
	%\logo{}
	%\institute{}
	%\date{}
	%\subject{}
	%\setbeamercovered{transparent}
	%\setbeamertemplate{navigation symbols}{}
	\begin{frame}[plain]
		\maketitle
	\end{frame}

\begin{frame}
	\frametitle{Themenüberblick}
	\tableofcontents
\end{frame}
	
\section{Mobbing}
\begin{frame}
\frametitle{Mobbing -- Formen des Mobbings}
\begin{center}
\begin{tikzpicture}[scale=0.8]
	\begin{scope}[blend group = soft light]
		\fill[red!30!white]   ( 90:1.2) circle (2);
		\fill[green!30!white] (210:1.2) circle (2);
		\fill[blue!30!white]  (330:1.2) circle (2);
	\end{scope}
	\node at ( 90:2)    {K};
	\node at ( 210:2)   {V};
	\node at ( 330:2)   {S};
\end{tikzpicture}
\end{center}
\begin{center}
	\textbf{K}örperliche Gewalt; \textbf{V}erbale Gewalt; \textbf{S}oziale Ausgrenzung 
\end{center}
\end{frame}

\begin{frame}\frametitle{Mobbing -- Formen des Mobbings}
\begin{description}[noitemsep]\setlength\itemsep{0.3em}
	\item[Phase 1:] Der Täter/die Täterin hat sich durch kleinere Gemeinheiten
	gegen verschiedene Kinder ein geeignetes Opfer ausgesucht.
	\item[Phase 2:] Systematische Attacken gegen das Opfer haben begonnen.
	\item[Phase 3:] Der Täter/die Täterin hat es geschafft, die Klasse zu
	überzeugen, so dass die Attacken gegenüber dem „Opfer“ als
	gerechtfertigt angesehen werden.\footnote{bzw. Opfer wird aus Angst nicht mehr unterstützt!}  
\end{description} 
	
\end{frame}

\begin{frame}
	\frametitle{Mobbing -- Formen}

\begin{description}[noitemsep]\setlength\itemsep{0.3em}
	\item[Körperliche Gewalt:] Dazu gehören Schlagen, Treten, Stoßen und andere Formen von körperlicher Gewalt.
	\item[Verbale Gewalt:] Dazu gehören Beschimpfen, Beleidigen, Bedrohen und andere Formen von verbaler Aggression.
	\item[Soziale Ausgrenzung:] Dazu gehört, jemanden aus der Gruppe auszuschließen, ihn zu ignorieren oder Gerüchte über ihn zu streuen.
	\item[Cybermobbing:] Dazu gehört das Senden von beleidigenden oder bedrohlichen Nachrichten, das Veröffentlichen von Fotos oder Videos des Opfers im Internet oder die Verbreitung von Gerüchten über das Opfer im Internet.
\end{description} 


\end{frame}

\section{Bundesrecht}
\begin{frame}
\frametitle{Bundesrecht}
\framesubtitle{Bildung in der Bundesverfassung}

\begin{description}[noitemsep]\setlength\itemsep{0.3em}
	\item[Art. 19] Der Anspruch auf ausreichenden und unentgeltlichen Grundschulunterricht ist
	gewährleistet.
	\item[Art. 62 Ziff. 1] Für das Bildungswesen sind die Kantone zuständig.
	\item[Art. 62 Ziff. 2] Sie sorgen für einen ausreichenden Grundschulunterricht, der allen Kindern offen steht. Der \textit{Grundschulunterricht ist obligatorisch} und untersteht staatlicher Leitung
	oder Aufsicht. \textit{An öffentlichen Schulen ist er unentgeltlich}.
\end{description} 
	
\end{frame}

\begin{frame}
	\frametitle{Bundesrecht}
	\framesubtitle{Schutz von Kindern/Jugendlichen in der Bundesverfassung}
	\begin{description}[noitemsep]\setlength\itemsep{0.3em}
		\item[Art. 11 Ziff. 1] Kinder und Jugendliche haben Anspruch auf besonderen Schutz ihrer Unversehrtheit und auf Förderung ihrer Entwicklung.
		\item[Art. 67 Ziff. 1]  Bund und Kantone tragen bei der Erfüllung ihrer Aufgaben den besonderen Förderungs- und Schutz- bedürfnissen von Kindern und Jugendlichen Rechnung.
	\end{description} 
	\vspace{2em}
	\underline{Hinweis:}\\
	\textit{Beinhaltet jedoch keine objektive einklagbare Verpflichtung sondern soll den Gesetzgeber ansprechen, sich dem Anliegen dieses Schutzes anzunehmen. Im Sinne des ZGB -- Elterliche Sorge}	\\
	{\footnotesize  G. Biaggini; Bundesverfassung der Schweizerischen Eidgenossenschaft; 2. Aufl. 2017; Art. 11 Rn. 5 \& Art. 67 Rn. 2}
	
\end{frame}

\end{document}